%% TeX source for "Micrometeoroid events in LISA Pathfinder" 
%% Author: I. Thorpe, NASA/GSFC
%% Note: Compiled using AASTeX 6.1

\documentclass[preprint, trackchanges]{aastex61}
%% options are:
%%  twocolumn   : two text columns, 10 point font, single spaced article.
%%                This is the most compact and represent the final published
%%                derived PDF copy of the accepted manuscript from the publisher
%%  manuscript  : one text column, 12 point font, double spaced article.
%%  preprint    : one text column, 12 point font, single spaced article.  
%%  preprint2   : two text columns, 12 point font, single spaced article.
%%  modern      : a stylish, single text column, 12 point font, article with
%% 		  wider left and right margins. This uses the Daniel
%% 		  Foreman-Mackey and David Hogg design.
%%
%% Note that you can submit to the AAS Journals in any of these 6 styles.
%%
%% There are other optional arguments one can envoke to allow other stylistic
%% actions. The available options are:
%%
%%  astrosymb    : Loads Astrosymb font and define \astrocommands. 
%%  tighten      : Makes baselineskip slightly smaller, only works with 
%%                 the twocolumn substyle.
%%  times        : uses times font instead of the default
%%  linenumbers  : turn on lineno package.
%%  trackchanges : required to see the revision mark up and print its output
%%  longauthor   : Do not use the more compressed footnote style (default) for 
%%                 the author/collaboration/affiliations. Instead print all
%%                 affiliation information after each name. Creates a much
%%                 long author list but may be desirable for short author papers
%%
%% these can be used in any combination, e.g.
%%
%% \documentclass[twocolumn,linenumbers,trackchanges]{aastex61}

%% Additional Packages
\usepackage{placeins}
\usepackage{color}

%% MACROs
\newcommand{\nwip}[2]{\langle #1|#2\rangle} % noise weighted inner product
\newcommand{\vdag}{(v)^\dagger}
\newcommand\aastex{AAS\TeX}
\newcommand\latex{La\TeX}
\newcommand{\red}[1]{\textcolor{red}{#1}}
\newcommand{\blue}[1]{\textcolor{blue}{#1}}
\newcommand{\nhits}{54 } % macro to keep number of impacts the same everywhere
\newcommand{\nhours}{4348 }

%% Dates
\received{\today}
\revised{\today}
\accepted{\today}
%% Command to document which AAS Journal the manuscript was submitted to.
%% Adds "Submitted to " the arguement.
\submitjournal{ApJ}

%% Mark up commands to limit the number of authors on the front page.
%% Note that in AASTeX v6.1 a \collaboration call (see below) counts as
%% an author in this case.
%
%\AuthorCollaborationLimit=7
%
%% Will only show Schwarz, Muench and "the AAS Journals Data Scientist 
%% collaboration" on the front page of this example manuscript.
%%
%% Note that all of the author will be shown in the published article.
%% This feature is meant to be used prior to acceptance to make the
%% front end of a long author article more manageable. Please do not use
%% this functionality for manuscripts with less than 20 authors. Conversely,
%% please do use this when the number of authors exceeds 40.
%%
%% Use \allauthors at the manuscript end to show the full author list.
%% This command should only be used with \AuthorCollaborationLimit is used.

%% The following command can be used to set the latex table counters.  It
%% is needed in this document because it uses a mix of latex tabular and
%% AASTeX deluxetables.  In general it should not be needed.
%\setcounter{table}{1}

%%%%%%%%%%%%%%%%%%%%%%%%%%%%%%%%%%%%%%%%%%%%%%%%%%%%%%%%%%%%%%%%%%%%%%%%%%%%%%%%
%%
%% The following section outlines numerous optional output that
%% can be displayed in the front matter or as running meta-data.
%%
%% If you wish, you may supply running head information, although
%% this information may be modified by the editorial offices.
\shorttitle{Micrometeoroid Impacts in LISA Pathfinder}
\shortauthors{Thorpe, et al.}
%%
%% You can add a light gray and diagonal water-mark to the first page 
%% with this command:
% \watermark{text}
%% where "text", e.g. DRAFT, is the text to appear.  If the text is 
%% long you can control the water-mark size with:
%  \setwatermarkfontsize{dimension}
%% where dimension is any recognized LaTeX dimension, e.g. pt, in, etc.
%%
%%%%%%%%%%%%%%%%%%%%%%%%%%%%%%%%%%%%%%%%%%%%%%%%%%%%%%%%%%%%%%%%%%%%%%%%%%%%%%%%

%% This is the end of the preamble.  Indicate the beginning of the
%% manuscript itself with \begin{document}.

\begin{document}

\title{Micrometeoroid Events in LISA Pathfinder}

%% LaTeX will automatically break titles if they run longer than
%% one line. However, you may use \\ to force a line break if
%% you desire. In v6.1 you can include a footnote in the title.

%% A significant change from earlier AASTEX versions is in the structure for 
%% calling author and affilations. The change was necessary to implement 
%% autoindexing of affilations which prior was a manual process that could 
%% easily be tedious in large author manuscripts.
%%
%% The \author command is the same as before except it now takes an optional
%% arguement which is the 16 digit ORCID. The syntax is:
%% \author[xxxx-xxxx-xxxx-xxxx]{Author Name}
%%
%% This will hyperlink the author name to the author's ORCID page. Note that
%% during compilation, LaTeX will do some limited checking of the format of
%% the ID to make sure it is valid.
%%
%% Use \affiliation for affiliation information. The old \affil is now aliased
%% to \affiliation. AASTeX v6.1 will automatically index these in the header.
%% When a duplicate is found its index will be the same as its previous entry.
%%
%% Note that \altaffilmark and \altaffiltext have been removed and thus 
%% can not be used to document secondary affiliations. If they are used latex
%% will issue a specific error message and quit. Please use multiple 
%% \affiliation calls for to document more than one affiliation.
%%
%% The new \altaffiliation can be used to indicate some secondary information
%% such as fellowships. This command produces a non-numeric footnote that is
%% set away from the numeric \affiliation footnotes.  NOTE that if an
%% \altaffiliation command is used it must come BEFORE the \affiliation call,
%% right after the \author command, in order to place the footnotes in
%% the proper location.
%%
%% Use \email to set provide email addresses. Each \email will appear on its
%% own line so you can put multiple email address in one \email call. A new
%% \correspondingauthor command is available in V6.1 to identify the
%% corresponding author of the manuscript. It is the author's responsibility
%% to make sure this name is also in the author list.
%%
%% While authors can be grouped inside the same \author and \affiliation
%% commands it is better to have a single author for each. This allows for
%% one to exploit all the new benefits and should make book-keeping easier.
%%
%% If done correctly the peer review system will be able to
%% automatically put the author and affiliation information from the manuscript
%% and save the corresponding author the trouble of entering it by hand.

%% Define the addresses (at least for the LPF collaboration)

\def\addressa{European Space Astronomy Centre, European Space Agency, Villanueva de la
Ca\~{n}ada, 28692 Madrid, Spain}
\def\addressb{Albert-Einstein-Institut, Max-Planck-Institut f\"ur Gravitationsphysik und Leibniz Universit\"at Hannover,
Callinstra{\ss}e 38, 30167 Hannover, Germany}
\def\addressc{APC, Univ Paris Diderot, CNRS/IN2P3, CEA/lrfu, Obs de Paris, Sorbonne Paris Cit\'e, France}
\def\addressd{High Energy Physics Group, Physics Department, Imperial College London, Blackett Laboratory, Prince Consort Road, London, SW7 2BW, UK }
\def\addresse{Dipartimento di Fisica, Universit\`a di Roma ``Tor Vergata'',  and INFN, sezione Roma Tor Vergata, I-00133 Roma, Italy}
\def\addressf{Department of Industrial Engineering, University of Trento, via Sommarive 9, 38123 Trento, 
and Trento Institute for Fundamental Physics and Application / INFN}
\def\addressg{Airbus Defence and Space, Claude-Dornier-Strasse, 88090 Immenstaad, Germany}
\def\addressh{European Space Technology Centre, European Space Agency, 
Keplerlaan 1, 2200 AG Noordwijk, The Netherlands}
\def\addressi{Dipartimento di Fisica, Universit\`a di Trento and Trento Institute for 
Fundamental Physics and Application / INFN, 38123 Povo, Trento, Italy}
\def\addressj{The School of Physics and Astronomy, University of
Birmingham, Birmingham, UK}
\def\addressk{Airbus Defence and Space, Gunnels Wood Road, Stevenage, Hertfordshire, SG1 2AS, UK }
\def\addressl{Institut f\"ur Geophysik, ETH Z\"urich, Sonneggstrasse 5, CH-8092, Z\"urich, Switzerland}
\def\addressm{The UK Astronomy Technology Centre, Royal Observatory, Edinburgh, Blackford Hill, Edinburgh, EH9 3HJ, UK}
\def\addressn{Institut de Ci\`encies de l'Espai (CSIC-IEEC), Campus UAB, Carrer de Can Magrans s/n, 08193 Cerdanyola del Vall\`es, Spain}
\def\addresso{DISPEA, Universit\`a di Urbino ``Carlo Bo'', Via S. Chiara, 27 61029 Urbino/INFN, Italy}
\def\addressp{European Space Operations Centre, European Space Agency, 64293 Darmstadt, Germany }
\def\addressq{Physik Institut, 
Universit\"at Z\"urich, Winterthurerstrasse 190, CH-8057 Z\"urich, Switzerland}
\def\addressr{SUPA, Institute for Gravitational Research, School of Physics and Astronomy, University of Glasgow, Glasgow, G12 8QQ, UK}
\def\addresss{Department d'Enginyeria Electr\`onica, Universitat Polit\`ecnica de Catalunya,  08034 Barcelona, Spain}
\def\addresst{Institut d'Estudis Espacials de Catalunya (IEEC), C/ Gran Capit\`a 2-4, 08034 Barcelona, Spain}
\def\addressu{Gravitational Astrophysics Lab, NASA Goddard Space Flight Center, 8800 Greenbelt Road, Greenbelt, MD 20771 USA}
\def\addressuuu{Code 674, NASA Goddard Space Flight Center, 8800 Greenbelt Road, Greenbelt, MD 20771 USA}
\def\addressx{INAF Osservatorio Astronomico di Capodimonte, I-80131 Napoli, Italy and INFN sezione di Napoli, I-80126 Napoli, Italy}
\def\addressxx{Dipartimento di Fisica, Universit\`a di Napoli ``Federico II'' and INFN -
Sezione di Napoli, I-80126, Napoli, Italy}
\def\addressy{INFN - Sezione di Napoli, I-80126, Napoli, Italy}
\def\addressz{Dipartimento di Fisica ed Astronomia, Universit\`a degli Studi di Firenze and INFN - Sezione di Firenze, I-50019 Firenze, Italy}
\def\addressuu{Center for Space Science \& Technology, University of Maryland Baltimore County, 1000 Hilltop Circle, Baltimore, Maryland 21250, USA  }
\def\addressaa{CGS S.p.A, Compagnia Generale per lo Spazio, Via Gallarate, 150 - 20151 Milano, Italy}

%% Corresponding Author

\correspondingauthor{J.I. Thorpe}
\email{james.i.thorpe@nasa.gov}

%% Primary Authors

\author[0000-0001-9276-4312]{J\,I~Thorpe}
\affiliation{\addressu}

\author{J~Slutsky}
\affiliation{\addressu}
\affiliation{\addressuu}

\author{John G. Baker}
\affiliation{\addressu}

\author{Tyson B. Littenberg}
\affiliation{NASA/MSFC}

\author{Nicole Pagane}
\affiliation{\addressu}
\affiliation{Johns Hopkins University}

\author{Sophie Hourihane}
\affiliation{NASA/MSFC}
\affiliation{NASA/GSFC}
\affiliation{University of Michigan}

\author{Petr Pokorny}
\affiliation{\addressuuu}
\affiliation{Catholic University of America}

\author{Diego Janches}
\affiliation{\addressuuu}
\nocollaboration

%% The LISA Pathfinder Collaboration
\collaboration{(The LISA Pathfinder Collaboration)}
\author{M~Armano}\affiliation{\addressa}
\author{H~Audley}\affiliation{\addressb}
\author{G~Auger}\affiliation{\addressc}
\author{J~Baird}\affiliation{\addressd}
\author{M~Bassan}\affiliation{\addresse}
\author{P~Binetruy}\altaffiliation{Deceased}\affiliation{\addressc}
\author{M~Born}\affiliation{\addressb}
\author{D~Bortoluzzi}\affiliation{\addressf}
\author{N~Brandt}\affiliation{\addressg}
\author{M~Caleno}\affiliation{\addressh}
\author{A~Cavalleri}\affiliation{\addressi}
\author{A~Cesarini}\affiliation{\addressi}
\author{A\,M~Cruise}\affiliation{\addressj}
\author{K~Danzmann}\affiliation{\addressb}
\author{M~de~Deus~Silva}\affiliation{\addressa}
\author{R~De~Rosa}\affiliation{\addressxx}
\author{L~Di~Fiore}\affiliation{\addressy}
\author{I~Diepholz}\affiliation{\addressb}
\author{G~Dixon}\affiliation{\addressj}
\author{R~Dolesi}\affiliation{\addressi}
\author{N~Dunbar}\affiliation{\addressk}
\author{L~Ferraioli}\affiliation{\addressl}
\author{V~Ferroni}\affiliation{\addressi}
\author{E\,D~Fitzsimons}\affiliation{\addressm}
\author{R~Flatscher}\affiliation{\addressg}
\author{M~Freschi}\affiliation{\addressa}
\author{C~Garc\'ia Marirrodriga}\affiliation{\addressh}
\author{R~Gerndt}\affiliation{\addressg}
\author{L~Gesa}\affiliation{\addressn}
\author{F~Gibert}\affiliation{\addressi}
\author{D~Giardini}\affiliation{\addressl}
\author{R~Giusteri}\affiliation{\addressi}
\author{A~Grado}\affiliation{\addressx}
\author{C~Grimani}\affiliation{\addresso}
\author{J~Grzymisch}\affiliation{\addressh}
\author{I~Harrison}\affiliation{\addressp}
\author{G~Heinzel}\affiliation{\addressb}
\author{M~Hewitson}\affiliation{\addressb}
\author{D~Hollington}\affiliation{\addressd}
\author{D~Hoyland}\affiliation{\addressj}
\author{M~Hueller}\affiliation{\addressi}
\author{H~Inchausp\'e}\affiliation{\addressc}
\author{O~Jennrich}\affiliation{\addressh}
\author{P~Jetzer}\affiliation{\addressq}
\author{B~Johlander}\affiliation{\addressh}
\author{N~Karnesis}\affiliation{\addressb}
\author{B~Kaune}\affiliation{\addressb}
\author{N~Korsakova}\affiliation{\addressb}
\author{C\,J~Killow}\affiliation{\addressr}
\author{J\,A~Lobo}\altaffiliation{Deceased}\affiliation{\addressn}
\author{I~Lloro}\affiliation{\addressn}
\author{L~Liu}\affiliation{\addressi}
\author{J\,P~L\'opez-Zaragoza}\affiliation{\addressn}
\author{R~Maarschalkerweerd}\affiliation{\addressp}
\author{D~Mance}\affiliation{\addressl}
\author{V~Mart\'{i}n}\affiliation{\addressn}
\author{L~Martin-Polo}\affiliation{\addressa}
\author{J~Martino}\affiliation{\addressc}
\author{F~Martin-Porqueras}\affiliation{\addressa}
\author{S\,Madden}\affiliation{\addressh}
\author{I~Mateos}\affiliation{\addressn}
\author{P\,W~McNamara}\affiliation{\addressh}
\author{J~Mendes}\affiliation{\addressp}
\author{L~Mendes}\affiliation{\addressa}
\author{M~Nofrarias}\affiliation{\addressn}
\author{S~Paczkowski}\affiliation{\addressb}
\author{M~Perreur-Lloyd}\affiliation{\addressr}
\author{A~Petiteau}\affiliation{\addressc}
\author{P~Pivato}\affiliation{\addressi}
\author{E~Plagnol}\affiliation{\addressc}
\author{P~Prat}\affiliation{\addressc}
\author{U~Ragnit}\affiliation{\addressh}
\author{J~Ramos-Castro}\affiliation{\addresss}
\author{J~Reiche}\affiliation{\addressb}
\author{D\,I~Robertson}\affiliation{\addressr}
\author{H\,Rozemeijer}\affiliation{\addressh}
\author{F~Rivas}\affiliation{\addressn}
\author{G~Russano}\affiliation{\addressi}
\author{P~Sarra}\affiliation{\addressaa}
\author{A~Schleicher}\affiliation{\addressg}
\author{D~Shaul}\affiliation{\addressd}
\author{C\,F~Sopuerta}\affiliation{\addressn}
\author{R~Stanga}\affiliation{\addressz}
\author{T~Sumner}\affiliation{\addressd}
\author{D~Texier}\affiliation{\addressa}
\author{C~Trenkel}\affiliation{\addressk}
\author{M~Tr{\"o}bs}\affiliation{\addressb}
\author{D~Vetrugno}\affiliation{\addressi}
\author{S~Vitale}\affiliation{\addressi}
\author{G~Wanner}\affiliation{\addressb}
\author{H~Ward}\affiliation{\addressr}
\author{P~Wass}\affiliation{\addressd}
\author{D~Wealthy}\affiliation{\addressk}
\author{W\,J~Weber}\affiliation{\addressi}
\author{L~Wissel}\affiliation{\addressb}
\author{A~Wittchen}\affiliation{\addressb}
\author{A~Zambotti}\affiliation{\addressf}
\author{C~Zanoni}\affiliation{\addressf}
\author{T~Ziegler}\affiliation{\addressg}
\author{P~Zweifel}\affiliation{\addressl}



%% Note that the \and command from previous versions of AASTeX is now
%% depreciated in this version as it is no longer necessary. AASTeX 
%% automatically takes care of all commas and "and"s between authors names.

%% AASTeX 6.1 has the new \collaboration and \nocollaboration commands to
%% provide the collaboration status of a group of authors. These commands 
%% can be used either before or after the list of corresponding authors. The
%% argument for \collaboration is the collaboration identifier. Authors are
%% encouraged to surround collaboration identifiers with ()s. The 
%% \nocollaboration command takes no argument and exists to indicate that
%% the nearby authors are not part of surrounding collaborations.

%% Mark off the abstract in the ``abstract'' environment. 
\begin{abstract}
The zodiacal dust complex, a population of dust and small particles that pervades the Solar System, provides important insight into the formation and dynamics of planets, comets, asteroids, and other bodies.  Efforts to understand this system have relied on analytic and numerical models anchored by observational data.  Much of this observational data is concentrated in the near-Earth regime, where the planet's gravitational effects mask some of the subtle but important differences between different models. Here we present a new set of data obtained using a novel technique: direct measurements of momentum transfer to a spacecraft from individual particle impacts. This technique is made possible by the extreme precision of the instruments flown on the LISA Pathfinder spacecraft, a technology demonstrator for a future space-based gravitational wave observatory that operated near the first Sun-Earth Lagrange point from early 2016 through Summer of 2017. Pathfinder employed a technique known as drag-free control whereby the spacecraft was commanded to maintain a constant position and attitude relative to a free-flying test mass within the instrument. This required rejection of external disturbances, including particle impacts, using a micropropulsion system. Using a simple model of the impacts and knowledge of the control system, we show that it is possible to detect impacts and measure properties such as the transferred momentum (related to the particle's mass and velocity), direction of travel, and location of impact on the spacecraft. In this paper, we present the results of a systematic search for impacts during \nhours hours of Pathfinder data. We report a total of \nhits candidates with momenta ranging from 0.2$\,\mu\textrm{Ns}$ to 230$\,\mu\textrm{Ns}$.  We furthermore make a comparison of these candidates with models of micrometeoroid populations in the inner solar system, those resulting from Jupiter-family comets, Oort-cloud comets, and Hailey-type comets. We find weak evidence suggesting that our observed population is most consistent with the properties of the Hailey-type comets although the statistics with this limited data set are modest.
\end{abstract}

%% Keywords should appear after the \end{abstract} command. 
%% See the online documentation for the full list of available subject
%% keywords and the rules for their use.
\keywords{dust, micrometeoroids --- 
miscellaneous}

%% From the front matter, we move on to the body of the paper.
%% Sections are demarcated by \section and \subsection, respectively.
%% Observe the use of the LaTeX \label
%% command after the \subsection to give a symbolic KEY to the
%% subsection for cross-referencing in a \ref command.
%% You can use LaTeX's \ref and \label commands to keep track of
%% cross-references to sections, equations, tables, and figures.
%% That way, if you change the order of any elements, LaTeX will
%% automatically renumber them.

%% We recommend that authors also use the natbib \citep
%% and \citet commands to identify citations.  The citations are
%% tied to the reference list via symbolic KEYs. The KEY corresponds
%% to the KEY in the \bibitem in the reference list below. 

\section{Introduction} \label{sec:intro}
Our Solar System hosts a population of dust and small particles that originate as debris from asteroids, comets, and other bodies.  Understanding these particles is important both for gaining insight into the formation of our Sun and its planets as well as for the dust population around other stars. More practically, dust and micrometeoroids are a critical component of the environment in which our spacecraft operate and against whose hazards they must be designed.  Like many fields in astrophysics, the behavior of the Solar System dust complex has been addressed from both theoretical and observational perspectives. Theorists have developed models of the production of dust from comets and asteroids,  its evolution under the effects of gravity and the solar environment, and its destruction through accretion and other processes. Observationally, this population has been constrained through measurements of its interaction with Earth's atmosphere (photographic, visual, and radio meteors, e.g. \cite{Halliday1984, Hawkes2007, Trigo-Rodriguez2008}), observations of zodiacal light (e.g.,  \cite{Krick2012, Durmont1980}), analysis of microcraters in Apollo lunar samples (e.g.  \cite{Allison1982}), and in-situ measurements made with ionization and penetration detectors on spacecraft (e.g.,  \cite{Weiden1978, Zhang1995}).  These theoretical and observational models are broadly consistent with one another, although important questions remain. One issue is that the bulk of the observational data is from the environment near Earth, a region in which some of the more subtle differences in the models of the underlying population are masked by the influence of the planet itself. Data taken far from Earth could in principle be used to distinguish such subtleties.
\\
LISA Pathfinder (LPF), A European Space Agency (ESA) mission which operated near the first Sun-Earth Lagrange point (L1) from January 2016 through July of 2017, is in an ideal orbit to make such measurements. However, LPF flew no instrumentation dedicated to micrometeoroid or dust detection.  LPF's primary objective was to demonstrate technologies for a future space-based observatory of milliHertz-band gravitational waves. The key achievement of LPF was placing two gold-platinum cubes known as `test masses' into a free-fall so pure it was characterized by accelerations at the femto-g level (e.g., \cite{LPF_PRL_2016, LPF_PRL_2018}), the level required to detect the minute disturbances caused by passing gravitational waves. In order to reach this level of performance, the test masses were released into cavities inside the spacecraft and a control system was employed to keep the spacecraft centered on the test masses.  This control system was designed to counteract disturbances on the spacecraft, including those caused by impacts from micrometeoroids. Shortly before LPF's launch, it was realized that data from the control system, if properly calibrated, could be used to detect and characterize these impacts and infer information about the impacting particles (e.g., \cite{Thorpe:2015cxa}). Early results from the first few months of LPF operations suggested that such events could indeed be identified and were roughly consistent with the pre-launch predictions of their effect on the control system (e.g., \cite{Thorpe2017a}). In this paper we present results from the first systematic search for micrometeoroid impacts in the LPF data set.  Our data set consists of \nhours hours of data in both the nominal LPF configuration as well as the "Disturbance Reduction System" (DRS) configuration, in which a NASA-supplied controller and thruster system took over control of the spacecraft. This data set corresponds to the times when LPF was operating in a `quiet' mode, without any intentional signal injections or other disturbances. During this period, we have identified \nhits impact candidates using our detection pipeline and manual vetoing. We characterized the properties of this data set and compared it to several theoretical models for the underlying dust population. 
\\
The remainder of the paper is organized as follows. In Section \ref{sec:models} we summarize the dust population models to which we compare our data set and their relevant properties. In Section \ref{sec:methods} we describe our detection technique, including initial calibration, search, parameter estimation, and vetoing. Section \ref{sec:results} summarizes our results, including examples of individual events, properties of the observed population, and statistical comparisons with the theoretical models. Conclusions from this work and implications for future work are contained in Section \ref{sec:conclusions}. A complete list of the impact candidates is included in Appendix A. 


\FloatBarrier
\section{Population Models}\label{sec:models}
Description of Models that we are looking at.  How are they motivated?  What are the differences between them. Describe their basic predictions for LPF orbit (rate, momentum distribution, and Sky Position). We should draft this and then get Petr/Diego to take a look at it and modify. 


See Figure \ref{fig:models}.

\begin{figure}
\gridline{\fig{figures/HTC.png}{0.3\textwidth}{(a) Halley-Type Comets (HTCs)}
          \fig{figures/JFC.png}{0.3\textwidth}{(b) Jupiter-Family Comets (JFCs)}
          \fig{figures/OCC.png}{0.3\textwidth}{(c) Oort Cloud Comets (OCCs)}}
\caption{Expected flux of micrometeoroid impacts as a function of impact direction for the environment around Sun-Earth L1 for three subpopulations of micrometeoroids. In these coordinates the sun is located at (0,0) and the prograde orbit of L1 is (-90,0). 
\label{fig:models}}
\end{figure}

\FloatBarrier
\section{Methods} \label{sec:methods}
The process of extracting micrometeoroid impact events from the LPF data stream can be divided into three distinct steps: calibration to equivalent free-body acceleration, detection and parameter estimation, and post-processing.  The following three subsections describe these three steps in more detail, the end result of which is a catalog of impact candidates.


\subsection{Calibration of LPF data}\label{sec:calibration}
As mentioned in the introduction, LPF uses a sophisticated control system to maintain the positions and attitudes of the spacecraft (S/C) and the two test masses (TMs) such that a number of constraints are satisfied.  Example constraints include maintaining the positions and orientations of the TMs at constant values relative to the S/C and maintaining the S/C attitude relative to the Sun and Earth. In total, the control system takes measurements of 15 kinematic degrees of freedom (DoFs), 3 positions + 3 attitudes for both test masses and 3 attitudes for the spacecraft, and generates actuation commands for 18 DoFs, 3 forces and 3 torques for the two test masses and the spacecraft.  Positions and angles are measured using a star tracker, a capacitive sensing system, and an optical interferometric sensing system. An electrostatic actuation system applies forces and torques to each TM and a micropropulsion system applies forces and torques to the S/C. 

\begin{figure}
\gridline{\fig{figures/rawImpact.eps}{0.5\textwidth}{(a) LPF Telemetry}
          \fig{figures/reconstructedImpact.eps}{0.5\textwidth}{(b) Equivalent free-body acceleration }}
\caption{Example of x-axis telemetry for impact candidate at GPS time XXX and the equivalent free-body acceleration estimated through the calibration procedure.\label{fig:models} \red{This is stolen from our LISA11 proceedings. We need to make a new version, maybe try the impact that we used for the first example of the MCMC results.}}
\end{figure}

One effect of the control system is to split the effect of a micrometeoroid impact into both the measured position and commanded force signals, both of which are telemetered down to ground. Figure \ref{fig:models}(a) shows an example of this for an impact candidate observed on 2016-05-16.  The top panel shows the measured position of one TM relative to the S/C along the x-axis, as measured using the optical interferometer.  For the first $\sim50\,$s, the signal exhibits random fluctuations with an RMS amplitude of a few nanometers. Just after 50$\,$s, the signal shows a steep upwards ramp, reaching more than $50\,$nm in a few seconds.  The bottom panel of Figure \ref{fig:models}(a) shows the force commands on the S/C in the x-direction, which are used to maintain the TM-S/C distance in this control mode. Shortly after the observed ramp in the motion, the controller commands a thrust of a few $\mu$N in the +x direction to compensate this motion.  The resulting acceleration of the S/C causes the TM-S/C separation to stop increasing, turn around, a return towards zero. In response, the controller reduces the applied force on the S/C. After two oscillations and roughly a minute, the system is back in its quiescent state.  By combining the force telemetry and the position telemetry with appropriate constants such as the calibration of the force actuators and the mass of the S/C, the equivalent free-body acceleration can be constructed. This is shown in Figure \ref{fig:models}(b) and exhibits the classic impulse response in acceleration that is expected for an impact. For this particular signal, which is one of the larger momentum impacts in our data set, a small overshoot in our reconstructed free-body acceleration can also be seen. This is likely to be due to errors in calibration factors or the relative timing of the various telemetries rather than an feature of the impact event itself, which occurs on timescales far shorter than the $0.1\,$s cadence of the telemetry. 

The basic process illustrated in Figure \ref{fig:models} can be repeated along the other DoFs of the S/C in order to develop a data set of the equivalent free-body acceleration of the S/C in all six DoFs. In doing so, a number of considerations must be addressed. First, the fact that the TMs are not located at the center of mass of the S/C means that torques applied to the S/C lead to accelerations in the linear DoFs of the S/C.  Secondly, the `topology' of the control system, or which actuations are used to control which displacements, is different for each DoF and also for the various operational modes of the control system. Lastly, generating the free-body accelerations requires knowledge of a number of calibration factors such as S/C and TM mass properties, location of the TMs in the S/C frame, locations of the thrusters on the spacecraft and their orientations, calibration and cross-talk in both sensors and actuators, and relative timing/phase information between the various telemetry.  Some of these effects and calibration factors were measured in-flight during dedicated experiments designed to calibrate the LPF hardware for its primary mission. Examples include calibration of the x-axis electrostatic TM actuation (e.g. \red{cal. paper}) and calibration of the thruster response (e.g. \red{ST7 paper, maybe Cold Gas paper}). For quantities that were not measured in flight, our models were built using the nominal values provided by the equipment manufacturers. 
\\
The end result of this calibration step was a set of twelve timeseries corresponding to the equivalent free-body acceleration of the S/C in each of 6 DoFs as measured by each of the two TMs. We denote these as $g_{1i}(t)$ and $g_{2i}(t)$ where $i=\left(x,\,y,\,z,\,\theta,\eta,\,\phi\right)$ for TM1 and TM2 respectively. The S/C coordinates are defined such that $z$ is the direction of the top deck (oriented at the Sun), $x$ is the direction along the two test masses with +x pointing from TM1 towards TM2, and $y$ completes a right-handed triad. The angles $\theta,\,\eta,\,\phi$ represent right-hand rotations around $x,\,y,\,z$ respectively.
\FloatBarrier

\subsection{Impact Model and Sensitivity}
\label{sec:sensitivity}
The characteristic timescales of the impact process are short relative the sample cadence of the LPF data (typically $0.1\,\textrm{s}$). Consequently, we model the impact as a delta-function impulse in acceleration for each DoF.  These impulses occur at the same time for each DoF but have different amplitudes which encode information about the impact direction and location on the spacecraft. The modeling of the impact is performed in two steps. First, the acceleration in the S/C body frame is computed for both linear and angular DoFs:
\begin{eqnarray}
\vec{a}_{x,B}(t) = P\:M^{-1}\delta(t-\tau) \hat{e},\label{eq:axB} \\
\vec{a}_{\theta}(t)= P\:\mathbf{I}^{-1}\delta(t-\tau)\left(\vec{r}\times\hat{e}\right), \label{eq:aq}
\end{eqnarray}
where $\vec{a}_{x,B}$ is the acceleration of the spacecraft body frame in the linear DoFs, $\vec{a}_{\theta,B}$ is the acceleration of the spacecraft body frame in the angular DoFs, $P$ is the total transferred momentum, $\tau$ is the impact time, $\hat{e}$ is the unit-vector in the direction of the transferred momentum, $M$ is the mass of the S/C, $\mathbf{I}$ is the S/C moment of inertia about its center of mass, and $\vec{r}$ is the location of the impact relative to the center of mass. The angular accelerations at the TM locations are the same as described in (\ref{eq:aq}) but the linear accelerations pick up an additional terms due to the offset of the test mass from the center of mass:
\begin{equation}
\vec{a}_{x,TM}(t) = \vec{a}_{x,B} + \left(\vec{r}_{TM}\times \vec{a}_{\theta}\right),\label{eq:axTM}
\end{equation}
where $\vec{a}_{x,TM}$ is the acceleration in the linear DoFs as measured in the test mass frame and $\vec{r}_{TM}$ is the location of the test mass relative to the S/C center of mass.

Sensitivity to impacts is limited by two noise sources: noise in the measurement system and disturbances on the S/C.  Measurement noise for both the capacitive and interferometric systems is characterized by a white spectrum in displacement whereas the chief noise source for the S/C disturbance, the micropropulsion system itself, exhibits an approximately white spectrum in force.  The relative levels of these two components differ for each DoF, but the basic functional form for the noise power spectral density is:
\begin{equation}
S_{g}=S_0+S_4f^4,
\label{eq:noise}
\end{equation}
where $S_0$ is the amplitude of the S/C disturbance term and $S_4$ is the amplitude of the measurement term. The most substantial difference between the noise level in the various DoFs is in the amplitude of the $S_4$ term, which is substantially lower for the DoFs sensed by the interferometric system: $x,\,\eta,$ and $\phi$.  In \cite{Thorpe:2015cxa} it was shown that SNR of a simple impulse in the presence of this noise shape can be analytically computed as $\rho =  P_i/P_c$, where $P_i$ is the amplitude of the momentum transfer in that DoF and $P_c$ is a characteristic threshold momentum given by:
\begin{equation}
P_c \equiv \frac{1}{\sqrt{2\pi}}\left(4 S_4 S_0^3\right)^{1/8}.\label{eq:SNRp}
\end{equation} 
The value of $P_c$ varies somewhat for each DoF due to the different combinations of sensing noise, micropropulsion noise, as well as differences in the spacecraft mass properties. The approximate range for linear DoFs is $0.05\,\mu\textrm{N-s}\leq P_c \leq 1\,\mu\textrm{N-s}$ and $0.3\,\mu\textrm{N-m-s}\leq P_c \leq 4\,\mu\textrm{N-m-s}$ for angular DoFs. This asymmetry in sensitivity along different DoFs means that impacts with lower overall momentum are often only detected in a fraction of DoF channels, meaning that the full set of parameters cannot be extracted. Similarly, impacts which happen to impart a large fraction of their momentum in a sensitive channel may be measured at lower thresholds than those coming from different directions.

\FloatBarrier
\subsection{Detection and Parameter Estimation}\label{sec:MCMC}
\red{Describe the MCMC tool and what tests we did on it. }


\FloatBarrier
\subsection{Post processing and vetos}\label{sec:vetos}
\red{Describe selection of candidates, application of vetos, and post-processing to do the sky angle conversion, sky area computation, etc.}
For each segment of data, the MCMC tool described in Section \ref{sec:MCMC} was run for in an initial search comprised of XXX steps on the TM1 data. After discarding the first XXX steps of the chain, the detection fraction was computed as the ratio of chain steps where an impact model was included to the total number of steps.  For systems with a detection fraction above XXX, the MCMC tool was re-run in a characterization step of XXX steps on both TM1 and TM2. A burn-in period of XXX steps was discarded from both chains and the detection fraction was again computed as well as the variance in the impact time parameter $\tau$.  For systems with an above-threshold detection fraction and impact time variance of less than XXX in both TMs were passed on to the next step in the vetting process, manual inspection.  For the manual inspection process, an expanded set of telemetry from the spacecraft around the candidate impact time was downloaded and inspected.  Examples of signals inspected include all force and torque signals, all position and attitude signals, selected voltage levels, and internal telemetry of the micropropulsion system. This process yielded two types of false triggers: thruster current spikes and data gaps. Candidates for which the signals appeared consistent with expectations were added to the catalog.

For vetted impact candidates, an additional post-processing step was conducted to extract parameters of interest.  In order to compare with the micrometeoroid population models in Section \ref{sec:models}, it was necessary to transform the impact direction from S/C coordinates to the \red{frame name?} used by the models.  This transformation was done in two steps, first from the S/C frame to an Earth-Centered Inertial (ECI) frame using the S/C quaternion telemetry provided by the star tracker, and then from ECI to the \red{frame name} using the S/C ephemeris. Median sky location and a 90\% confidence sky area for both frames was computed using HEALPIX.\red{cite something}.

\FloatBarrier
\section{Results} \label{sec:results}
In this paper we restrict our analysis to segments of data where no signals were deliberately injected into the LPF system. We identified a total of \nhours hours of data in three distinct configurations: the nominal LPF configuration in which the European-provided DFACS control system and cold gas micropropulsion system were operating (2957 hours), the DRS configuration in which the NASA-provided DCS control system and colloidal micropropulsion system were operating (1207 hours), and a hybrid configuration in which the DFACS was controlling the S/C using the colloidals (184 hours). Figure \ref{fig:timeline} shows a timeline of these segments along with the detected impacts plotted with their total transferred momentum along the vertical axis. The total number of detected impacts is 54: 36 in the nominal configuration, 15 in the DRS configuration, and 3 in the hybrid configuration. This corresponds to a rough event rate of 0.3 impacts/day, which is broadly consistent with the estimate made in \cite{Thorpe:2015cxa}. 

\begin{figure}[ht!]
\plotone{figures/timeline.png}
\caption{Timeline of impact events during LPF. The yellow dots show the impact times with the total transferred momentum defining the vertical axis. The vertical bars denote the times included in the search with blue representing the nominal configuration, red the DRS configuration, and purple the hybrid configuration. See text for details. \label{fig:timeline}}
\end{figure}

In the following sections we present some example events in detail and summarize some properties of the observed population.  A full catalog of the impacts and their estimated parameters can be found in Appendix A.

\FloatBarrier
\subsection{Sample candidate events \label{sec:samples}}
As mentioned in Section \ref{sec:sensitivity}, LPF's sensitivity depends on the parameters of the impact including both the total momentum transferred as well as the fraction of that momentum that is  projected into each DoF.  As a result, the quality of our parameter estimation varies greatly from impact to impact.  Figure \ref{fig:goodExample} shows results for an impact occurring at GPS time $t_{gps}=1154024345.4$, corresponding to 2016-07-31 18:18:48.400 UTC. With a moderately high transferred momentum of $8.5\,\mu\textrm{N-s}$ and a S/C longitude that aligns well with the sensitive x-axis, the total SNR in the two TMs are $\rho_1\approx16$ and $\rho_2\approx?$ . Figure \ref{fig:goodExample}(a) shows an overlaid corner plot representing the posterior probabilities for the impact parameters as measured by TM1 (in red, lower-left) and TM2 (in blue, upper-right).  The panels are arranged in a grid with rows and columns corresponding to the following parameters: total transferred momentum ($P_{tot}$, in $\mu$N-s), S/C latitude defined relative to the S/C x-y plane ($lat$, in deg.), S/C longitude defined relative to the +x axis ($lon$, in deg.), and x,y,z, locations of the impact with respect to the S/C center of mass ($r_x,\,r_y,\,r_z$ in $m$). \red{are the locations with respect to mechanical frame or body frame?}. The panels along the diagonal show the posterior probability density for each parameter as measured by TM1 (red) and TM2 (blue).  The panels on the off-diagonals show the correlation between pairs of parameters in the TM1 data (lower off-diagonals) and TM2 data (upper off-diagonals). The measured parameters between these two impacts are broadly consistent, although TM1 generally prefers a solution with slightly increased $P_{tot}$, larger $lat$, and positive shifts in both $r_x$ and $r_z$. The variances in the parameters for TM1 are also slightly smaller than for TM2, likely as a result of increased signal in TM1.\red{we should check, I think we have SNR data for both}.

Figure \ref{fig:goodExample}(b) shows the reconstructed sky localization in the \red{frame name} frame using the TM1 data.  The 90\% confidence interval \red{(check)} encompasses an area of 3860$\,\textrm{deg}^2$. and is centered at a latitude of -43 deg and a longitude of -91 deg, meaning that the impactor originated from the direction of LPF's orbit and from below the ecliptic. Figure \ref{fig:goodExample}(c) shows the reconstructed S/C impact location on a map of the LPF exterior that has been flattened to show the full surface. \red{we need some labels on these figures to make these easier to understand}.  The most likely location of impact is on the +y vertical panel about 2/3 of the way up towards the top deck. \red{we should check the table program because the table shows +x+y}.


\begin{figure}[h!]
\gridline{\fig{figures/dualCorner.png}{0.3\textwidth}{(a) Parameter corner plot for GRS1 and GRS2}
          \fig{figures/sky_sun2.png}{0.3\textwidth}{(b) Sky localization in Sun frame}
          \fig{figures/flat_LPF_log_GRS1.png}{0.3\textwidth}{(c) Spacecraft Location}}
\caption{Sample MCMC results for a well-characterized event.\label{fig:goodExample}}
\end{figure}

Figure \ref{fig:Example} shows a similar set of plots as Figure \ref{fig:goodExample} but for an impact occurring at $t_{gps} = 1149475987.7$ (2016-06-09 02:52:50.700 UTC) which had a lower total momentum ($P_{tot}\approx1.0\,\mu\textrm{N-s}$), and lower SNR ($\rho_1\approx1.5$, $\rho_2\approx?$). As a result, the constraints on parameters other than the total momentum are rather weak.  The impact location is favored towards the -x and +z faces and the preferred direction to the impactor is in the forward direction (latitudes around -90 deg) and above the ecliptic.
\begin{figure}[h!]
\gridline{\fig{figures/dualCorner_low.png}{0.3\textwidth}{(a) Parameter corner plot for GRS1 and GRS2}
          \fig{figures/sky_sun1_low.png}{0.3\textwidth}{(b) Sky localization in Sun frame}
          \fig{figures/flat_LPF_log_GRS1_low.png}{0.3\textwidth}{(c) Spacecraft Location}}
\caption{Sample MCMC results for a typical event.\label{fig:Example}}
\end{figure}

\FloatBarrier
\subsection{Ensemble results}
Before comparing our observations with the detailed models of the dust complex described section \ref{sec:models}, it is useful to examine some basic properties of the observed ensemble. Figure \ref{fig:CDF_P} shows the cumulative momentum distribution of the observed events, with the median momentum indicated by the yellow diamond \red{marker} and the \red{XXX} confidence interval indicated by the horizontal bars.  As is generally expected from all models of the dust complex, the number of impacts falls with the total momentum. Two single-index power-law fits to the data, both inlcuding and excluding the highest-momentum impact, are plotted in the red and purple lines, and have power-law indices of -0.73 and -0.97 respectively. \red{check numbers and get error bars}. Also plotted for reference are the predicted flux averaged over the four models described in section \ref{sec:models} and a fit to them with a power law index of \red{XXX}. \red{What else do we want to say here? Do we want to try fitting various breaks, etc.  Is there a nice Bayesian way to do that?}.

\begin{figure}
\gridline{\fig{figures/momentumCDF.png}{0.5\textwidth}{a}}
\caption{Cumulative momentum distribution of observed impacts. We should divide by the number of impacts to get this in a fractional sense. \label{fig:CDF_P}}
\end{figure}

A second property of the observed distribution is the time of arrival statistics, which are expected to be governed by a Poisson process. Figure \ref{fig:CDF_rate} shows the cumulative probability density of the observed time between events, which was computed by excising the times not included in our data set. As is expected for a Poisson process, this distribution follows an exponential function, with a time between events of $2.94\pm0.05\,$days. 
\begin{figure}
\gridline{\fig{figures/rateplot.eps}{0.5\textwidth}{a}}
\caption{Cumulative distribution of observed time interval between events, taking into account gaps in the observations. The red curve is an exponential fit with a characteristic interval between events of $2.94\pm0.05\,$days. \label{fig:CDF_rate}}
\end{figure}


\FloatBarrier
\section{Model Inference} \label{sec:models}

Describe how we did the model inference and what the results are. Maybe grab some of John's notes.
\begin{figure}
\gridline{\fig{figures/JFC_HTC.png}{0.3\textwidth}{(a) JFC-HTC subpopulation}
          \fig{figures/HTC_Uniform.png}{0.3\textwidth}{(b) HTC-Uniform subpopulation}
          \fig{figures/Uniform_JFC.png}{0.3\textwidth}{(c) Uniform-JFC subpopulation}}
\caption{Likelihood distribution for mixed population model based on the LPF impact data set. The preferred point corresponds to an approximately equal mixture of HTC and Uniform populaitons.  JFC is disfavored at the XXX level, blah, blah...\label{fig:Example}}
\end{figure}
\FloatBarrier
\section{Conclusions} \label{sec:conclusions}

\begin{itemize}
\item Micrometeoroids are important and we want to know more
\item We used an entirely novel method to sample the micrometeoroid population at L1
\item We are consistent (or not) with models and we have some idea as to why (or why not)
\item Some rough statements about what this means for LISA  and also other missions (telescopes)
\end{itemize}
%% If you wish to include an acknowledgments section in your paper,
%% separate it off from the body of the text using the \acknowledgments
%% command.
\acknowledgments

This analysis was conducted under a 2017 NASA Science Innovation Fund awarded to Thorpe, Littenberg, Janches, and Baker. Slutsky acknowledges support of the NASA Astrophysics Division. Pagane and Hourihane acknowledge the support of the NASA Undergraduate Summer Internship program and Hourihane acknowledges the support of the National Science Foundation's Research Experience for Undergraduates program.
\\

The data was produced by the LISA Pathfinder mission, which was part of the
space-science programme of the European Space Agency and also hosted the NASA Disturbance Reduction System payload, developed under the NASA New Millennium Program. 

The French contribution to LISA Pathfinder has been supported by the CNES (Accord Specific de projet
CNES 1316634/CNRS 103747), the CNRS, the Observatoire de Paris and the University
Paris-Diderot. E.~Plagnol and H.~Inchausp\'{e} would also like to acknowledge the
financial support of the UnivEarthS Labex program at Sorbonne Paris Cit\'{e}
(ANR-10-LABX-0023 and ANR-11-IDEX-0005-02).

The Albert-Einstein-Institut acknowledges the support of the German Space Agency,
DLR, in the development and operations of LISA Pathfinder. The work is supported by the Federal Ministry for Economic Affairs and Energy
based on a resolution of the German Bundestag (FKZ 50OQ0501 and FKZ 50OQ1601). 

The Italian contribution to LISA Pathfinder has been supported  by Agenzia Spaziale Italiana and Istituto
Nazionale di Fisica Nucleare.

The Spanish contribution to LISA Pathfinder has been supported by contracts AYA2010-15709 (MICINN),
ESP2013-47637-P, and ESP2015-67234-P (MINECO). M.~Nofrarias acknowledges support from
Fundacion General CSIC (Programa ComFuturo). F.~Rivas acknowledges an FPI contract
(MINECO).

The Swiss contribution to LISA Pathfinder was made possible by the support of the Swiss Space Office (SSO)
via the PRODEX Programme of ESA. L.~Ferraioli is supported by the Swiss National
Science Foundation.

The UK LISA Pathfinder groups wish to acknowledge support from the United Kingdom Space Agency
(UKSA), the University of Glasgow, the University of Birmingham, Imperial College,
and the Scottish Universities Physics Alliance (SUPA).

%% To help institutions obtain information on the effectiveness of their 
%% telescopes the AAS Journals has created a group of keywords for telescope 
%% facilities.
%
%% Following the acknowledgments section, use the following syntax and the
%% \facility{} or \facilities{} macros to list the keywords of facilities used 
%% in the research for the paper.  Each keyword is check against the master 
%% list during copy editing.  Individual instruments can be provided in 
%% parentheses, after the keyword, but they are not verified.

\vspace{5mm}
\facilities{LPF (\url{http://sci.esa.int/lisa-pathfinder/})}

%% Similar to \facility{}, there is the optional \software command to allow 
%% authors a place to specify which programs were used during the creation of 
%% the manusscript. Authors should list each code and include either a
%% citation or url to the code inside ()s when available.

\software{LTPDA (\url{https://www.elisascience.org/ltpda/})}

%% Appendix material should be preceded with a single \appendix command.
%% There should be a \section command for each appendix. Mark appendix
%% subsections with the same markup you use in the main body of the paper.

%% Each Appendix (indicated with \section) will be lettered A, B, C, etc.
%% The equation counter will reset when it encounters the \appendix
%% command and will number appendix equations (A1), (A2), etc. The
%% Figure and Table counter will not reset


 \appendix
 \section{List of Impact Events in LPF}
The catalog of LPF impact events is reported in the table below. For each impact, the date of impact, GPS timestamp, and median inferred momentum transfer with 95\% confidence intervals are listed. For impacts with an inferred 95\% confidence sky location of less than 4100 $deg^2$ (10\% of the sky), the 95\% error area as well as the impact direction in both spacecraft and Sun Synchronous Ecliptic Coordinates is reported. For impacts with a greater than 75\% probability of impacting on a particular face of the spacecraft, the spacecraft face is identified. The location of LPF in its orbit at the time of the impact is provided in EME2000 (J2000) coordinates. 
			\begingroup
			\renewcommand\arraystretch{2}
			\begin{longtable*}{|c|c|c|c|c|c|c|c|c|c|c|c|}

				\hline 
				& 
				& 
				& 
				&
				&
				\multicolumn{4}{|c|}{Sky Location} &
				\multicolumn{3}{|c|}{LPF Position (EME2000)} \\
				\multicolumn{1}{|c}{Date} & 
				\multicolumn{1}{|c|}{GPS [s]}  & 
				\multicolumn{1}{|c|}{$P_{med}$} & 
				\multicolumn{1}{|c|}{Face} &
				\multicolumn{1}{|c|}{Localization} &
				\multicolumn{1}{|c|}{$Lat_{SC}$} &
				\multicolumn{1}{|c|}{$Lon_{SC} $} &
				\multicolumn{1}{|c|}{$Lat_{SSE}$} &
				\multicolumn{1}{|c|}{$Lon_{SSE}$} &
				\multicolumn{1}{|c|}{$X$} &
				\multicolumn{1}{|c|}{$Y$} &
				\multicolumn{1}{|c|}{$Z$} \\
				& 
				[s] & 
				$[\mu N s]$& 
				&
				$[deg^2]$&
				\multicolumn{1}{|c|}{$[deg]$} &
				\multicolumn{1}{|c|}{$ [deg]$} &
				\multicolumn{1}{|c|}{$[deg]$} &
				\multicolumn{1}{|c|}{$[deg]$} &
				\multicolumn{1}{|c|}{$[Mkm]$} &
				\multicolumn{1}{|c|}{$[Mkm]$} &
				\multicolumn{1}{|c|}{$[Mkm]$} \\
				\hline
			\endfirsthead
			        
				\multicolumn{12}{r}{continued from previous page} \\
				\hline 
				& 
				& 
				& 
				&
				&
				\multicolumn{4}{|c|}{Sky Location} &
				\multicolumn{3}{|c|}{LPF Position (EME2000)} \\
				\multicolumn{1}{|c}{Date} & 
				\multicolumn{1}{|c|}{GPS [s]}  & 
				\multicolumn{1}{|c|}{$P_{med}$} & 
				\multicolumn{1}{|c|}{Face} &
				\multicolumn{1}{|c|}{Localization} &
				\multicolumn{1}{|c|}{$Lat_{SC}$} &
				\multicolumn{1}{|c|}{$Lon_{SC} $} &
				\multicolumn{1}{|c|}{$Lat_{SSE}$} &
				\multicolumn{1}{|c|}{$Lon_{SSE}$} &
				\multicolumn{1}{|c|}{$X$} &
				\multicolumn{1}{|c|}{$Y$} &
				\multicolumn{1}{|c|}{$Z$} \\
				& 
				[s] & 
				$[\mu N s]$& 
				&
				$[deg^2]$&
				\multicolumn{1}{|c|}{$[deg]$} &
				\multicolumn{1}{|c|}{$ [deg]$} &
				\multicolumn{1}{|c|}{$[deg]$} &
				\multicolumn{1}{|c|}{$[deg]$} &
				\multicolumn{1}{|c|}{$[Mkm]$} &
				\multicolumn{1}{|c|}{$[Mkm]$} &
				\multicolumn{1}{|c|}{$[Mkm]$} \\	
				\hline		
				\endhead
			
			\hline 
			\multicolumn{12}{|r|}{{Continued on next page}} \\ 
			\hline
			\endfoot

			\hline
			\endlastfoot
	2016-04-09 & 1144229908 & $17.2^{+0.4}_{-0.3}$ & +y+y & 1729 & -7 & -7 & -57 & -39 & 1.09 & 0.55 & -0.05 \\
	2016-05-04 & 1146429822 & $ 1.7^{+3.1}_{-0.6}$ & - & - & - & - & - & - & 0.45 & 1.24 & 0.56 \\
	2016-05-16 & 1147442122 & $ 0.7^{+0.5}_{-0.5}$ & - & - & - & - & - & - & 0.14 & 1.34 & 0.77 \\
	2016-05-16 & 1147453726 & $14.4^{+0.8}_{-0.4}$ & +x+x & 3438 & -2 & 162 & 45 & -56 & 0.14 & 1.34 & 0.77 \\
	2016-05-19 & 1147693044 & $ 0.9^{+0.9}_{-0.3}$ & - & - & - & - & - & - & 0.08 & 1.35 & 0.81 \\
	2016-05-19 & 1147741578 & $ 2.0^{+0.8}_{-0.3}$ & - & - & - & - & - & - & 0.06 & 1.35 & 0.82 \\
	2016-06-08 & 1149475988 & $ 1.0^{+1.1}_{-0.3}$ & - & - & - & - & - & - & -0.26 & 1.36 & 1.00 \\
	2016-06-20 & 1150511110 & $ 3.5^{+1.7}_{-1.2}$ & - & - & - & - & - & - & -0.35 & 1.33 & 1.03 \\
	2016-07-07 & 1151901050 & $ 0.2^{+0.5}_{-0.1}$ & - & - & - & - & - & - & -0.42 & 1.35 & 1.00 \\
	2016-07-24 & 1153404058 & $ 2.9^{+1.3}_{-0.3}$ & - & - & - & - & - & - & -0.48 & 1.37 & 0.87 \\
	2016-07-28 & 1153750663 & $19.9^{+1.7}_{-1.3}$ & +z+z & 2585 & 18 & 158 & -31 & -172 & -0.51 & 1.38 & 0.83 \\
	2016-07-31 & 1154024345 & $ 8.6^{+1.8}_{-1.6}$ & +x+y & 3857 & -7 & 128 & -7 & 156 & -0.54 & 1.38 & 0.79 \\
	2016-08-11 & 1154963503 & $ 2.4^{+0.8}_{-0.3}$ & - & - & - & - & - & - & -0.66 & 1.35 & 0.64 \\
	2016-08-17 & 1155461605 & $ 0.5^{+1.3}_{-0.3}$ & - & - & - & - & - & - & -0.73 & 1.31 & 0.54 \\
	2016-08-18 & 1155558407 & $ 1.6^{+1.3}_{-0.8}$ & - & - & - & - & - & - & -0.74 & 1.30 & 0.52 \\
	2016-08-19 & 1155637974 & $12.1^{+3.0}_{-3.2}$ & +z+z & 1786 & 68 & -87 & -4 & -39 & -0.76 & 1.29 & 0.50 \\
	2016-08-19 & 1155677822 & $ 2.4^{+1.7}_{-2.3}$ & +z+z & - & - & - & - & - & -0.76 & 1.29 & 0.50 \\
	2016-08-22 & 1155891413 & $ 0.7^{+0.3}_{-0.2}$ & - & - & - & - & - & - & -0.80 & 1.26 & 0.45 \\
	2016-08-23 & 1155985559 & $23.8^{+2.6}_{-2.1}$ & +z+z & 84 & 87 & -112 & -7 & -58 & -0.82 & 1.25 & 0.43 \\
	2016-08-23 & 1156020427 & $ 0.9^{+3.0}_{-0.7}$ & +z+z & - & - & - & - & - & -0.83 & 1.25 & 0.42 \\
	2016-08-24 & 1156063801 & $ 1.0^{+0.7}_{-0.8}$ & - & - & - & - & - & - & -0.83 & 1.24 & 0.41 \\
	2016-08-24 & 1156115516 & $ 3.0^{+1.0}_{-0.9}$ & +z+z & 1873 & 77 & -105 & -11 & -53 & -0.84 & 1.23 & 0.40 \\
	2016-08-25 & 1156188047 & $ 0.5^{+1.2}_{-0.4}$ & - & - & - & - & - & - & -0.86 & 1.22 & 0.39 \\
	2016-08-26 & 1156255314 & $ 0.6^{+2.8}_{-0.3}$ & - & - & - & - & - & - & -0.87 & 1.21 & 0.37 \\
	2016-09-15 & 1157966718 & $ 1.1^{+1.3}_{-0.4}$ & - & - & - & - & - & - & -1.14 & 0.70 & -0.04 \\
	2016-10-05 & 1159736213 & $ 0.9^{+1.5}_{-0.6}$ & +z+z & - & - & - & - & - & -1.15 & -0.18 & -0.41 \\
	2016-10-06 & 1159808666 & $230.3^{+4.8}_{-5.8}$ & +x+y & 430 & 4 & 101 & -62 & 116 & -1.14 & -0.21 & -0.42 \\
	2016-10-07 & 1159869088 & $ 6.4^{+2.8}_{-3.4}$ & +z+z & 2645 & 66 & 3 & -18 & 6 & -1.13 & -0.25 & -0.43 \\
	2016-12-02 & 1164719570 & $ 0.6^{+0.6}_{-0.3}$ & - & - & - & - & - & - & 0.06 & -1.62 & -0.29 \\
	2016-12-20 & 1166268578 & $ 8.0^{+3.1}_{-2.8}$ & - & - & - & - & - & - & 0.20 & -1.65 & -0.23 \\
	2016-12-21 & 1166337501 & $ 1.6^{+1.1}_{-0.4}$ & - & - & - & - & - & - & 0.21 & -1.65 & -0.23 \\
	2016-12-26 & 1166805122 & $ 0.5^{+1.1}_{-0.4}$ & - & - & - & - & - & - & 0.23 & -1.65 & -0.24 \\
	2016-12-27 & 1166921605 & $28.6^{+1.2}_{-0.9}$ & +y-x & 1716 & 19 & 13 & -12 & -91 & 0.24 & -1.65 & -0.24 \\
	2016-12-28 & 1166995369 & $ 0.8^{+0.9}_{-0.3}$ & - & - & - & - & - & - & 0.25 & -1.64 & -0.25 \\
	2017-01-01 & 1167307196 & $22.5^{+0.8}_{-0.7}$ & +x+x & 2149 & -7 & 150 & 17 & 114 & 0.26 & -1.64 & -0.26 \\
	2017-01-04 & 1167613479 & $ 0.9^{+1.0}_{-0.3}$ & - & - & - & - & - & - & 0.28 & -1.62 & -0.28 \\
	2017-01-05 & 1167654180 & $10.3^{+2.1}_{-1.5}$ & - & - & - & - & - & - & 0.28 & -1.62 & -0.28 \\
	2017-01-08 & 1167944728 & $ 4.5^{+0.6}_{-0.3}$ & -y-y & - & - & - & - & - & 0.30 & -1.61 & -0.30 \\
	2017-01-10 & 1168061759 & $ 3.5^{+0.9}_{-0.7}$ & -y-y & - & - & - & - & - & 0.30 & -1.60 & -0.31 \\
	2017-01-12 & 1168267680 & $ 1.2^{+1.0}_{-0.3}$ & - & - & - & - & - & - & 0.31 & -1.59 & -0.33 \\
	2017-02-12 & 1170979672 & $ 1.8^{+1.2}_{-0.4}$ & - & - & - & - & - & - & 0.68 & -1.22 & -0.60 \\
	2017-02-13 & 1171012017 & $ 2.5^{+2.2}_{-1.1}$ & - & - & - & - & - & - & 0.68 & -1.22 & -0.60 \\
	2017-03-11 & 1173291241 & $ 1.9^{+0.9}_{-0.3}$ & - & - & - & - & - & - & 1.12 & -0.38 & -0.55 \\
	2017-04-22 & 1176914535 & $ 1.0^{+1.0}_{-0.3}$ & - & - & - & - & - & - & 0.76 & 1.10 & 0.43 \\
	2017-04-22 & 1176917343 & $ 1.1^{+1.2}_{-0.3}$ & - & - & - & - & - & - & 0.76 & 1.10 & 0.43 \\
	2017-05-04 & 1177956916 & $ 1.3^{+1.1}_{-0.3}$ & - & - & - & - & - & - & 0.48 & 1.27 & 0.70 \\
	2017-05-05 & 1178035038 & $40.2^{+5.8}_{-6.6}$ & -y+x & 168 & -83 & -63 & -43 & -91 & 0.46 & 1.27 & 0.72 \\
	2017-05-06 & 1178120384 & $ 1.5^{+1.0}_{-0.3}$ & - & - & - & - & - & - & 0.44 & 1.28 & 0.73 \\
	2017-05-07 & 1178197245 & $ 1.2^{+1.1}_{-0.3}$ & - & - & - & - & - & - & 0.43 & 1.29 & 0.75 \\
	2017-05-08 & 1178251226 & $11.7^{+0.9}_{-0.3}$ & -y-y & - & - & - & - & - & 0.41 & 1.29 & 0.76 \\
	2017-05-18 & 1179167273 & $14.0^{+3.9}_{-2.5}$ & +x+y & 3015 & 8 & 84 & 27 & -142 & 0.23 & 1.33 & 0.93 \\
	2017-05-22 & 1179493289 & $ 8.0^{+0.8}_{-0.5}$ & +y-x & 3864 & -1 & 25 & -27 & -173 & 0.18 & 1.34 & 0.97 \\
	2017-06-04 & 1180613326 & $ 1.2^{+1.5}_{-0.4}$ & - & - & - & - & - & - & 0.02 & 1.33 & 1.08 \\
	2017-06-11 & 1181272382 & $ 1.0^{+1.0}_{-0.3}$ & - & - & - & - & - & - & -0.02 & 1.33 & 1.11 \\
	\hline
\end{longtable*} 
\endgroup


%% The reference list follows the main body and any appendices.
%% Use LaTeX's thebibliography environment to mark up your reference list.
%% Note \begin{thebibliography} is followed by an empty set of
%% curly braces.  If you forget this, LaTeX will generate the error
%% "Perhaps a missing \item?".
%%
%% thebibliography produces citations in the text using \bibitem-\cite
%% cross-referencing. Each reference is preceded by a
%% \bibitem command that defines in curly braces the KEY that corresponds
%% to the KEY in the \cite commands (see the first section above).
%% Make sure that you provide a unique KEY for every \bibitem or else the
%% paper will not LaTeX. The square brackets should contain
%% the citation text that LaTeX will insert in
%% place of the \cite commands.

%% We have used macros to produce journal name abbreviations.
%% \aastex provides a number of these for the more frequently-cited journals.
%% See the Author Guide for a list of them.

%% Note that the style of the \bibitem labels (in []) is slightly
%% different from previous examples.  The natbib system solves a host
%% of citation expression problems, but it is necessary to clearly
%% delimit the year from the author name used in the citation.
%% See the natbib documentation for more details and options.

\bibliography{Bibliography2}

%% This command is needed to show the entire author+affilation list when
%% the collaboration and author truncation commands are used.  It has to
%% go at the end of the manuscript.
%\allauthors

%% Include this line if you are using the \added, \replaced, \deleted
%% commands to see a summary list of all changes at the end of the article.
\listofchanges

\end{document}

